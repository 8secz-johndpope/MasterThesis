\chapter{Proposed Method}
この章がメイン.提案手法について説明する.

ここが短いと卒業(修了)させてもらえないらしい.

アルゴリズムの入れ方のサンプルをAlgorithm~\ref{alg:prop}に貼っときます.
\begin{algorithm}[tbp]
 \caption{Proposed method using CSPF}\label{alg:prop}
 \begin{algorithmic}[1]
  \Require $C$:生成回路となるAQFP論理回路
  \State $C \gets$ RAND-1を用いてAQFP論理回路に変換された入力回路
  \State コメント \Comment{ステップ1}
  \State 行にもラベルを貼れます(参照テスト[\ref{alg:prop:hoge}行目]) \label{alg:prop:hoge}

  \If{あほあほ}
  \State ifはこんな感じ
  \State \Return \True
  \ElsIf{ぼけぼけ}
  \State \Return \False
  \EndIf

  \For{$i=1$ to N}
  \State forはもちろん
  \EndFor

  \ForAll{外部出力リストのゲート\textit{po}}
  \State forallや
  \EndFor

  \Repeat
  \State リピートも使えます
  \State $C$内のそれぞれのゲートの出力論理を計算
  \State OptimizeCircuitUsingCSPF($C$)
  \Until{$C$の構成に変更なし}

  \State \Return $C$
 \end{algorithmic}
\end{algorithm}
