\chapter{Introduction}
\par Seiler claims that "The presentation is the art of persuasion. It plays a significant role in our society and has the tremendous impact on the success of everyone"\cite{seiler2002communication}. However, it is not as simple as computer data transmission which requires certain skills\cite{rosenberg2005acoustic}. One aspect is the preparation of the slides and speech draft in advance, and it can be relatively easily refined by revising the presentation plan and the speech draft many times by oneselves or professionals. The other is the way the orator delivers the presentation to others, and it contains verbal style and various nonverbal behaviors. 
\par The important component of a presentation lies upon nonverbal cues which have the power to change the meaning assigned to spoken words\cite{seiler2002communication}. According to Argyle nonverbal messages are thirteen to fourteen times more powerful than verbal ones\cite{argyle1971communication}. Likewise, Arcy showed that the audience receives more than half of information from nonverbal behaviors\cite{d1998communicating}.  The fact that most people unconsciously believe more in nonverbal behaviors than verbal cues was also found during the study of Seiler\cite{seiler2002communication}.
\par Unfortunately, it's hard to practice to express the effective nonverbal behaviors because they are mostly expressed subconsciously. According to Seiler \cite{seiler2002communication}, imitate past famous speech will help trainee to refine their nonverbal behaviors. To refine the trainees' nonverbal behaviors, we propose a presentation training system that allow trainee to imitate past famous speech to improve their nonverbal behaviors. Nonverbal behaviors include many aspects, and we choose to analyze the motion of orators as the nonverbal behaviors in this paper. We employed OpenPose library\cite{cao2017realtime} to extract orators' motion data from past famous speech 2D video. While training, the system capture trainee's motion in real-time using Microsoft Kinect and match it with the motion in the past famous speech. We choose the cosine similarity of adjacent limbs as features to get the similarity of the trainees' motion and the motion of extracted orators.

