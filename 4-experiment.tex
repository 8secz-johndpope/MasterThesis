\chapter{Evaluation}
本章では,提案した○○手法を○○に適用した実験結果とその考察を示す.

\section{Method}
評価方法が細かい説明が要る場合は,節を分けて方法だけを説明するのもあり.
節を分けるほどでなければ,実験したマシン,環境,などについて1から2段落で説明する.

\section{Results}

結果を表\ref{table:result_bench1}に示す.

表の例:たての罫線(外側)を使わないのが格好いいらしい.

\begin{table}[tbp]
\caption{分析結果1}
\begin{tabular}{l|r|r|r|r} \Hline
 & \multicolumn{1}{l|}{gcc} & \multicolumn{1}{l|}{li} & \multicolumn{1}{l|}{m88ksim} & \multicolumn{1}{l}{go} \\\hline
収束回数 & 4346  & 5930  & 2595  & 3545  \\\hline
総保存命令数 & 50711  & 72519  & 28562  & 81305  \\\hline
再利用可能な総命令数 & 47366  & 59178  & 28277  & 78317  \\\hline
平均再利用可能命令数 & 10.90  & 12.30  & 10.90  & 22.09  \\\hline
保存した命令が一部, 再利用できない回数 & 363  & 1350  & 35  & 349  \\\hline
\end{tabular}
\label{table:result_bench1}
\end{table}


以上の結果より以下のことが言える.
実験結果の後は,考察を分かりやすくまとめる.箇条書きを使ってもよい.

*すぐに導かれる将来の課題はここに書いてしまっても良い.そして,「おわりに」に
再度書いてもOK
